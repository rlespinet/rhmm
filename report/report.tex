\documentclass[a4paper, 11pt]{article}

\usepackage[a4paper,margin=1in]{geometry}
\usepackage[english]{babel}
\usepackage[utf8]{inputenc}
\usepackage[T1]{fontenc}
\usepackage{lmodern}
\usepackage{listings}
\usepackage{graphicx}
\usepackage{amsmath}
\usepackage{amssymb}
\usepackage{framed}
\usepackage{amsfonts}
\usepackage{caption}
\usepackage{subcaption}
\usepackage{listings}
\usepackage{tabularx}
\usepackage{hyperref}
\usepackage{color}
\usepackage[dvipsnames]{xcolor}
\usepackage{fancyhdr}
\usepackage{lastpage}
% \usepackage[round, sort]{natbib}
\usepackage{tikz}

\bibliographystyle {abbrv}
\usetikzlibrary{decorations.pathreplacing, matrix}

\graphicspath{{../imgs/}}

\definecolor{morange}{RGB}{237,106,90}
\definecolor{mgreen}{RGB}{63,127,95}
\definecolor{mpurple}{RGB}{127,0,85}

\lstset{
  basicstyle=\small\ttfamily, % Global Code Style
  captionpos=b, % Position of the Caption (t for top, b for bottom)
  extendedchars=true, % Allows 256 instead of 128 ASCII characters
  tabsize=2, % number of spaces indented when discovering a tab
  columns=fixed, % make all characters equal width
  keepspaces=true, % does not ignore spaces to fit width, convert tabs to spaces
  showstringspaces=false, % lets spaces in strings appear as real spaces
  breaklines=true, % wrap lines if they don't fit
  frame=trbl, % draw a frame at the top, right, left and bottom of the listing
  frameround=tttt, % make the frame round at all four corners
  framesep=4pt, % quarter circle size of the round corners
  numbers=left, % show line numbers at the left
  numberstyle=\tiny\ttfamily, % style of the line numbers
  commentstyle=\color{mgreen}, % style of comments
  keywordstyle=\color{mpurple}, % style of keywords
  stringstyle=\color{morange}, % style of strings
}

% TAILLE DES PAGES (A4 serré)

\setlength{\intextsep}{2em}
\setlength{\parindent}{0pt}
\setlength{\parskip}{1em}
%% \setlength{\textwidth}{17cm}
%% \setlength{\textheight}{24cm}
%% \setlength{\oddsidemargin}{-.7cm}
%% \setlength{\evensidemargin}{-.7cm}
%% \setlength{\topmargin}{-.5in}


\pagestyle{fancy}
\renewcommand{\headrulewidth}{0pt}
\renewcommand{\footrulewidth}{0.6pt}% default is 0pt
\lhead{}
\rhead{}
\lfoot{Page \thepage\ of \pageref{LastPage}}
\rfoot{Rémi Lespinet}
\cfoot{}
\cfoot{}

\newcounter{cquestion}
\renewcommand{\thecquestion}{\arabic{cquestion}}
\newenvironment{question}
{\par \vspace{0.5em} \noindent \stepcounter{cquestion} \hspace{-1em}
 $\bullet$ \underline{Q\thecquestion :}}
{}

\newenvironment{note}
{\begin{framed} \textbf{Note : }}
{\end{framed}}

\newenvironment{comment}
% {\color{gray} \textbf{Comment : }}
{\color{gray}}
{}


% Commandes de mise en page
\newcommand{\file}[1]{\lstinline{#1}}
\newcommand{\name}[1]{\emph{#1}}
\newcommand{\Fig}[1]{Fig \ref{#1} p. \pageref{#1}}
\newcommand{\Figure}[1]{Figure \ref{#1} p. \pageref{#1}}
\newcommand{\Tab}[1]{Tab \ref{#1} p. \pageref{#1}}
\newcommand{\Table}[1]{Table \ref{#1} p. \pageref{#1}}
\newcommand{\itemi}{\item[$\bullet$]}
% Commandes color
\newcommand{\colgood}[1]{\color{ForestGreen} #1}
\newcommand{\colbad}[1]{\color{BrickRed} #1}


% Commandes de maths
\newcommand{\function}[3]{#1 : #2 \to #3}
\newcommand{\intn}[2]{\left\{ #1 \dots #2 \right\}}
\newcommand{\intr}[2]{\left[ #1 ; #2 \right]}
\newcommand{\intro}[2]{\left] #1 ; #2 \right[}
\newcommand{\dotp}[2]{\langle #1, #2 \rangle}
\newcommand{\logn}[1]{\ln\left( #1\right)}
%% \newcommand{\det}[1]{\left| #1 \right|}
\newcommand{\pd}[2]{\frac{\partial #1}{\partial #2}}
\newcommand{\grad}{\nabla}
\newcommand{\norm}[1]{\|#1\|}
\newcommand{\set}[2]{\left\{ #1 \hspace{.5em} ; \hspace{.5em}#2 \right\}}
\newcommand{\eqdef}{\triangleq}
\newcommand{\tr}[1]{Tr\left( #1 \right)}
\newcommand{\lse}[1]{\text{lse}\left( #1 \right)}
\newcommand{\pcond}[2]{p(#1 \hspace{-.2em}\mid\hspace{-.2em} #2)}
\newcommand{\e}[1]{\mathop{\mathbb{E}}\left[#1\right]}
\newcommand{\conde}[2]{\mathop{\mathbb{E}}\left[#1 \hspace{.2em}\mid\hspace{.2em} #2 \right]}
\newcommand{\ssemi}{\hspace{.3em};\hspace{.3em}}



\newcommand{\iid}{i.i.d }
\newcommand{\wrt}{w.r.t }

% Commandes informatique
\newcommand{\pfun}[1]{{\textbf{\texttt{#1}}}}

\newcommand{\ipart}[1]{\vspace{0.5em}\textbf{#1}\vspace{0.5em}}



\pagenumbering{arabic}

\title{\textsc{Reinforcement learning - MVA 2017/2018 \\ \emph{Internship report}} }
\author{Rémi Lespinet}
\date{}

\begin{document}

\maketitle
\thispagestyle{fancy}


\section{Abstract}

This report presents my work as an intern in the NPI team
(Neuropsychologie Interventionnelle)

\section{Introduction}

The goal was to try to predict dementia in the population based on
data related to aging \url{https://g2aging.org/}.

Dementia is a medical condition that affects the ability to think and
remember and is severe enough to reduce a person's ability to perform
daily tasks. It is a group of symptoms rather than a specific disease
which causes the affected person to lose its autonomy (most of the
time, the person must get helped from a caregiver).

In the context of the internship, we are interested in two tasks :
\begin{itemize}
\item Classify dement individuals based on the available features, and
  determine which features have the highest impact on the prediction.
\item Predict the evolution of dementia for a given individual
\end{itemize}
% \begin{itemize}
% \item We want to classify dement individuals based on the
%   available features, and determine which features have the highest
%   impact on the prediction.
% \item As an ultimate objective, we are also interested in predicting
%   the evolution of dementia for a given individual
% \end{itemize}
Dementia is a very important topic nowadays and will be of increasing
interest in the upcoming years, due to ageing of the population,
particularly in emerging countries \cite{?}.

\section{Dataset}

The Gateway to global aging (\url{https://g2aging.org/}) is an
initiative sponsored by the National Institute on Aging to facilitate
cross-national comparative studies on aging. It contains informations
on

\begin{itemize}
\item Health, cognition, diseases, emotion, injury, physical
functioning, physical measures, health behaviors
\item Health insurance, health services utilization, health care
expenditure, out-of-pocket spending
\item Labor force, employment status/history, earnings,
disability, retirement, work characteristics, pensions
\item Economic Status, income, wealth, consumption,
earnings, government transfers, housing, financial and
non-financial assets
\item Family structure, social network, family demographics,
family exchange, family support, marriage, social
participation
\end{itemize}

\section{Problem presentation}

\subsection{Context}

\subsection{Data}

\subsection{Literature}

% Hurd & Laurent

\section{Data preprocessing}

% dead still answering

\section{Missing value imputation}

\subsection{Probabilistic PCA}

% Impute PCA

\section{Feature selection on Adams}

\subsection{LASSO}

\subsection{Gradient boosting algorithms}

\section{Unsupervised learning}

\subsection{Agglomerative clustering}

\subsection{Gaussian mixtures models with missing data}

\section{Semi-supervised learning}

\subsection{Transductive SVMs}

\section{Hidden markov models}

\subsection{Fully unsupervised HMM}

\subsection{Semi-supervised HMM}

\section{Trajectory clustering}

% KMLShape

\section{Prediction}

\subsection{LSTM for prediction}


\section{Weighted batched update}

Let
\begin{displaymath}
  \xi_k^l = \sum_{i = k}^{l} p_i
\end{displaymath}

\subsection{Mean}

\begin{displaymath}
  \mu_{n + k} = \dfrac{\sum_{i = 1}^{n+k} p_i x_i}{\sum_{i = 1}^{n+k} p_i}
    % \mu_{n + k} = \dfrac{\sum\limits_{i = 1}^{n+k} p_i x_i}{\sum\limits_{i = 1}^{n+k} p_i}
\end{displaymath}

% \begin{align*}
%   \mu_{n + k}
%   &= \dfrac{1}{\sum_{i = 1}^{n+k} p_i}
%     \left[
%     \sum_{i = 1}^{n} p_i x_i +
%     \sum_{i = n+1}^{n+k} p_i x_i
%     \right] \\
% \end{align*}

% \begin{align*}
%   \mu_{n + k} - \mu_{n}
%   &= \dfrac{1}{\xi_{1}^{n+k}}
%     \left[
%     \sum_{i = 1}^{n} p_i x_i +
%     \sum_{i = n+1}^{n+k} p_i x_i
%     \right] -
%     \dfrac{1}{\xi_{1}^{n}}
%     \sum_{i = 1}^{n} p_i x_i \\
% \end{align*}


\begin{align*}
  \mu_{n + k}
  &=
    \dfrac{1}{\xi_1^{n+k}} \sum_{i = 1}^{n+k} p_i x_i \\
  &=
    \dfrac{1}{\xi_{1}^{n+k}}
    \left[
    \sum_{i = 1}^{n} p_i x_i +
    \sum_{i = n+1}^{n+k} p_i x_i
    \right] \\
  &= \dfrac{\xi_{1}^{n}}{\xi_{1}^{n+k}}
    \mu_n +
    \dfrac{1}{\xi_{1}^{n+k}}
    \sum_{i = n+1}^{n+k} p_i x_i \\
  &= \dfrac{\xi_{1}^{n+k} - \xi_{n+1}^{n+k}}{\xi_{1}^{n+k}}
    \mu_n +
    \dfrac{1}{\xi_{1}^{n+k}}
    \sum_{i = n+1}^{n+k} p_i x_i \\
  &= \mu_n +
    \dfrac{1}{\xi_{1}^{n+k}}
    \sum_{i = n+1}^{n+k} p_i (x_i - \mu_n) \\
\end{align*}
Hence,
\begin{framed}
  \begin{displaymath}
    \mu_{n+k}= \mu_n +
    \dfrac{\sum_{i = n+1}^{n+k} p_i (x_i - \mu_n)}{\sum_{1}^{n+k}p_i}
  \end{displaymath}
\end{framed}
Additionally, in log scale,
\begin{displaymath}
  \log{\mu_{n+k}}= \lse{
    \mu_n,
    \log{\sum_{n+1}^{n+k} p_i(x_i - \mu_n)}
    - \log{\sum_{n+1}^{n+k} p_i(x_i - \mu_n)}
  }
\end{displaymath}
\begin{displaymath}
  \log{\mu_{n+k}}= \lse{
    \mu_n,
    \lse{\left\{\log{p_i} + \log{(x_i - \mu_n)} \right\}_{n+1}^{n+k} }
    - \lse{\left\{\log{p_i} + \log{(x_i - \mu_n)} \right\}_{1}^{n+k} }
  }
\end{displaymath}


% \begin{align*}
%   \mu_{n + k}
%   &= \dfrac{1}{\sum_{1}^{n+k}p_i}
%     \left[
%     \sum_{i = 1}^{n} p_i x_i +
%     \sum_{i = n+1}^{n+k} p_i x_i
%     \right] \\
%   &= \dfrac{\sum_{1}^{n}p_i}{\sum_{1}^{n+k}p_i}
%     \mu_n +
%     \dfrac{1}{\sum_{1}^{n+k}p_i}
%     \sum_{i = n+1}^{n+k} p_i x_i \\
%   &= \dfrac{\sum_{1}^{n+k}p_i - \sum_{n+1}^{n+k}p_i}{\sum_{1}^{n+k}p_i}
%     \mu_n +
%     \dfrac{1}{\sum_{1}^{n+k}p_i}
%     \sum_{i = n+1}^{n+k} p_i x_i \\
%   &= \mu_n +
%     \dfrac{1}{\sum_{1}^{n+k}p_i}
%     \sum_{i = n+1}^{n+k} p_i (x_i - \mu_n) \\
% \end{align*}


% \begin{align*}
%   \mu_{n + k}
%   &= \dfrac{1}{\pi_{n+k}}
%     \left[
%     \sum_{i = 1}^{n} p_i x_i +
%     \sum_{i = n+1}^{n+k} p_i x_i
%     \right] \\
%   &= \dfrac{\pi_{n}}{\pi_{n+k}}
%     \mu_n +
%     \dfrac{1}{\pi_{n+k}}
%     \sum_{i = n+1}^{n+k} p_i x_i \\
%   &= \dfrac{\pi_{n+k} - \sum_{n+1}^{n+k}p_i}{\pi_{n+k}}
%     \mu_n +
%     \dfrac{1}{\pi_{n+k}}
%     \sum_{i = n+1}^{n+k} p_i x_i \\
%   &= \mu_n +
%     \dfrac{1}{\pi_{n+k}}
%     \sum_{i = n+1}^{n+k} p_i (x_i - \mu_n) \\
% \end{align*}


\subsection{Covariance}

\begin{align*}
  R_n^{n+k}
  &=
    \xi_{1}^{n + k} \Sigma_{n + k} - \xi_{1}^{n} \Sigma_{n}\\
  &=
    \sum_{i = 1}^{n+k} p_i (x_i - \mu_{n+k}) (x_i - \mu_{n+k})^T -
    \sum_{i = 1}^{n} p_i (x_i - \mu_{n}) (x_i - \mu_{n})^T \\
  &=
    \sum_{i = n+1}^{n+k} p_i (x_i - \mu_{n+k}) (x_i - \mu_{n+k})^T +
    \sum_{i = 1}^{n} p_i
    \left[
    (x_i - \mu_{n+k}) (x_i - \mu_{n+k})^T -
    (x_i - \mu_{n}) (x_i - \mu_{n})^T
    \right]
\end{align*}
For all $i$, we have
\begin{align*}
  C_{n}^{n+k}(i)
  &=
    (x_i - \mu_{n+k}) (x_i - \mu_{n+k})^T -
    (x_i - \mu_{n}) (x_i - \mu_{n})^T \\[1em]
  &=
    (x_i - \mu_{n} + \mu_{n} - \mu_{n+k}) (x_i - \mu_{n} + \mu_{n} - \mu_{n+k})^T -
    (x_i - \mu_{n}) (x_i - \mu_{n})^T \\[1em]
  &=
    (x_i - \mu_{n} ) (\mu_{n} - \mu_{n+k})^T + (\mu_{n} - \mu_{n+k}) (x_i - \mu_{n})^T + (\mu_{n} - \mu_{n+k}) (\mu_{n} - \mu_{n+k})^T
\end{align*}
Since
\begin{displaymath}
  \sum_{i=1}^{n} p_i (x_i - \mu_n) =  0
\end{displaymath}
We have
\begin{align*}
  \sum_{i=1}^{n} p_i C_n^{n+k}(i)
  &=
    \sum_{i=1}^{n} p_i  (\mu_{n} - \mu_{n+k}) (\mu_{n} - \mu_{n+k})^T
\end{align*}
Hence
\begin{align*}
  R_n^{n+k}
  &=
    \sum_{i = n+1}^{n+k} p_i (x_i - \mu_{n+k}) (x_i - \mu_{n+k})^T +
    \sum_{i=1}^{n} p_i
    (\mu_{n} - \mu_{n+k}) (\mu_{n} - \mu_{n+k})^T \\
  &=
    \sum_{i = n+1}^{n+k} p_i (x_i - \mu_{n+k}) (x_i - \mu_{n+k})^T +
    \left(\sum_{i=1}^{n} p_i \mu_{n} - \sum_{i=1}^{n} p_i \mu_{n+k} \right)
     (\mu_{n} - \mu_{n+k})^T \\
  &=
    \sum_{i = n+1}^{n+k} p_i (x_i - \mu_{n+k}) (x_i - \mu_{n+k})^T +
    \left(\sum_{i=1}^{n} p_i \mu_{n} - \sum_{i=1}^{n+k} p_i \mu_{n+k} + \sum_{i=n+1}^{n+k} p_i \mu_{n+k} \right)
     (\mu_{n} - \mu_{n+k})^T \\
  &=
    \sum_{i = n+1}^{n+k} p_i (x_i - \mu_{n+k}) (x_i - \mu_{n+k})^T +
    \left(\sum_{i=1}^{n} p_i x_i - \sum_{i=1}^{n+k} p_i x_i + \sum_{i=n+1}^{n+k} p_i \mu_{n+k} \right)
     (\mu_{n} - \mu_{n+k})^T \\
  &=
    \sum_{i = n+1}^{n+k} p_i (x_i - \mu_{n+k}) (x_i - \mu_{n+k})^T +
    \left(
    \sum_{i=n+1}^{n+k} p_i \mu_{n+k} -
    \sum_{i=n+1}^{n+k} p_i x_i
    \right)
     (\mu_{n} - \mu_{n+k})^T \\
  &=
    \sum_{i = n+1}^{n+k} p_i (x_i - \mu_{n+k}) (x_i - \mu_{n+k})^T +
    \sum_{i = n+1}^{n+k} p_i (x_i - \mu_{n+k}) (\mu_{n+k} - \mu_{n})^T \\
  &=
    \sum_{i = n+1}^{n+k} p_i (x_i - \mu_{n+k}) (x_i - \mu_{n})^T \\
\end{align*}
Finally,
\begin{framed}
\begin{displaymath}
  \xi_{1}^{n + k} \Sigma_{n + k}
  = \xi_{1}^{n} \Sigma_{n} +
  \sum_{i = n+1}^{n+k} p_i (x_i - \mu_{n+k}) (x_i - \mu_{n})^T \\
\end{displaymath}
% \begin{displaymath}
%   \xi_{1}^{n + k} \Sigma_{n + k}
%   = \xi_{1}^{n} \Sigma_{n} +
%     \sum_{i = n+1}^{n+k} p_i (x_i - \mu_{n+k}) (x_i - \mu_{n+k})^T +
%     \xi_{1}^{n}
%     (\mu_{n} - \mu_{n+k}) (\mu_{n} - \mu_{n+k})^T \\
% \end{displaymath}
\end{framed}


% \begin{align*}
%   \mu_{n + k}
%   &= \dfrac{1}{\xi_{1}^{n+k}}
%     \left[
%     \sum_{i = 1}^{n} p_i x_i +
%     \sum_{i = n+1}^{n+k} p_i x_i
%     \right] \\
%   &= \dfrac{\xi_{1}^{n}}{\xi_{1}^{n+k}}
%     \mu_n +
%     \dfrac{1}{\xi_{1}^{n+k}}
%     \sum_{i = n+1}^{n+k} p_i x_i \\
%   &= \dfrac{\xi_{1}^{n+k} - \xi_{n+1}^{n+k}}{\xi_{1}^{n+k}}
%     \mu_n +
%     \dfrac{1}{\xi_{1}^{n+k}}
%     \sum_{i = n+1}^{n+k} p_i x_i \\
%   &= \mu_n +
%     \dfrac{1}{\xi_{1}^{n+k}}
%     \sum_{i = n+1}^{n+k} p_i (x_i - \mu_n) \\
    %   \end{align*}



\section{Semi-supervised HMM with mixed states}

\begin{displaymath}
  p(y, q) = p(y_0 \dots y_T, q_0 \dots q_T) = p(q_0) \prod_{t = 0}^{T-1} p(q_{t+1} \mid q_t) \prod_{t = 0}^{T} p(y_t \mid q_t)
\end{displaymath}

\begin{displaymath}
  \mathcal{L}(\theta \ssemi y, q) = \log{p(y, q \ssemi \theta)} = \log{p(q_0)} + \sum_{t = 0}^{T-1} \log{p(q_{t+1} \mid q_t)} + \sum_{t = 0}^{T} \log{ p(y_t \mid q_t) }
\end{displaymath}

\begin{displaymath}
  \mathcal{L}(\theta \ssemi y, q) = \sum_{i = 1}^{M} q_0^i \log{\pi_i} + \sum_{t = 0}^{T-1} \sum_{i = 1}^{M} \sum_{j = 1}^{M} q_t^i q_{t+1}^j \log{a_{i,j}} + \sum_{t = 0}^{T} \sum_{i = 1}^{M} q_t^i \log{ p(y_t \mid q_t^i) }
\end{displaymath}

\begin{displaymath}
  \mathcal{L}(\theta \ssemi (y_1, q_1) \dots (y_N, q_N) ) = \sum_{n = 1}^N \sum_{i = 1}^{M} q_{n, 0}^i \log{\pi_i} + \sum_{n = 1}^N \sum_{t = 0}^{T-1} \sum_{i = 1}^{M} \sum_{j = 1}^{M} q_{n, t}^i q_{n, t+1}^j \log{a_{i,j}} + \sum_{n = 1}^N \sum_{t = 0}^{T} \sum_{i = 1}^{M} q_{n, t}^i \log{ p(y_{n, t} \mid q_{n, t}^i) }
\end{displaymath}


% \begin{displaymath}
%   \mathcal{L}(\theta \ssemi (y_1, q_1) \dots (y_N, q_N) ) = \sum_{n = 1}^N \sum_{i = 1}^{M} q_{n, 0}^i \log{\pi_i} + \sum_{n = 1}^N \sum_{t = 0}^{T-1} \sum_{i = 1}^{M} \sum_{j = 1}^{M} q_{n, t}^i q_{n, t+1}^j \log{a_{i,j}} + \sum_{n = 1}^N \sum_{t = 0}^{T} \sum_{i = 1}^{M} q_{n, t}^i \log{ p(y_{n, t} \mid q_{n, t}^i) }
% \end{displaymath}


\begin{displaymath}
  l_{n, t} = \left\{
    \begin{array}{ll}
      1 &\text{ if $q_{n, t}$ is observed } \\
      0 &\text{ otherwise }
    \end{array}
    \right.
\end{displaymath}


% \begin{displaymath}
%   \mathcal{L}(\theta \ssemi (y_1, q_1) \dots (y_N, q_N) )
%   = \sum_{n = 1}^N \sum_{i = 1}^{M} q_{n, 0}^i \log{\pi_i}
%   + \sum_{n = 1}^N \sum_{t = 0}^{T-1} \sum_{i = 1}^{M} \sum_{j = 1}^{M} q_{n, t}^i q_{n, t+1}^j \log{a_{i,j}}
%   + \sum_{n = 1}^N \sum_{t = 0}^{T} \sum_{i = 1}^{M} q_{n, t}^i \log{ p(y_{n, t} \mid q_{n, t}^i \ssemi \theta_i) }
% \end{displaymath}

\begin{displaymath}
  \begin{split}
    \mathcal{L}(\theta \ssemi (y_1, q_1) \dots (y_N, q_N) )
    = &\sum_{n = 1}^N \sum_{i = 1}^{M} q_{n, 0}^i \log{\pi_i}
    + \sum_{n = 1}^N \sum_{t = 0}^{T-1} \sum_{i = 1}^{M} \sum_{j = 1}^{M} q_{n, t}^i q_{n, t+1}^j \log{a_{i,j}} \\
    & + \sum_{n = 1}^N \sum_{t = 0}^{T} \sum_{i = 1}^{M} q_{n, t}^i \log{ p(y_{n, t} \mid q_{n, t}^i \ssemi \theta_i) }
  \end{split}
\end{displaymath}


% \begin{displaymath}
%   \begin{split}
%     \mathcal{L}(\theta \ssemi (y_1, q_1) \dots (y_N, q_N) )
%     & = \sum_{n = 1}^N \sum_{i = 1}^{M} l_{n, 0} q_{n, 0}^i  \log{\pi_i} +
%     \sum_{n = 1}^N \sum_{t = 0}^{T-1} \sum_{i = 1}^{M} \sum_{j = 1}^{M} l_{n, t} q_{n, t}^i q_{n, t+1}^j \log{a_{i,j}} \\
%     & + \sum_{n = 1}^N \sum_{t = 0}^{T} \sum_{i = 1}^{M} l_{n, t} q_{n, t}^i \log{ p(y_{n, t} \mid q_{n, t}^i) } \\
%     & + \sum_{n = 1}^N \sum_{i = 1}^{M} (1 - l_{n, 0}) q_{n, 0}^i  \log{\pi_i}
%     \sum_{n = 1}^N \sum_{t = 0}^{T-1} \sum_{i = 1}^{M} \sum_{j = 1}^{M} (1 - l_{n, t}) q_{n, t}^i q_{n, t+1}^j \log{a_{i,j}}\\
%     & + \sum_{n = 1}^N \sum_{t = 0}^{T} \sum_{i = 1}^{M} (1 - l_{n, t}) q_{n, t}^i \log{ p(y_{n, t} \mid q_{n, t}^i) }
%   \end{split}
% \end{displaymath}

\begin{displaymath}
  \conde{q_{n, t}^i \log{ p(y_{n, t} \mid q_{n, t}^i)}}{y_{n, t}, l_{n, t}} =
  \left(
    l_{n, t} \cdot q_{n, t}^i + \overline{l_{n, t}} \cdot p(q_{n, t}^i = 1 \mid y)
  \right)
  \log{ p(y_{n, t} \mid q_{n, t}^i \ssemi \theta_i)}
\end{displaymath}

\begin{displaymath}
  \begin{split}
    \conde{ q_{n, t}^i q_{n, t+1}^j \log{a_{i,j}}}{y_{n}, l_{n}}
    & = l_{n, t} \cdot l_{n, t+1} \cdot q_{n, t}^i \cdot q_{n, t+1}^j \log{a_{i, j}}\\
    & + l_{n, t} \cdot \overline{l_{n, t+1}} \cdot q_{n, t}^i \cdot p(q_{n, t+1}^j = 1 \mid y_{n}) \log{a_{i, j}} \\
    & + \overline{l_{n, t}} \cdot l_{n, t+1} \cdot p(q_{n, t}^i = 1 \mid y_{n}) \cdot q_{n, t + 1}^j \log{a_{i, j}} \\
    & + \overline{l_{n, t}} \cdot \overline{l_{n, t+1}} \cdot p(q_{n, t}^i = 1 \wedge q_{n, t + 1}^j = 1  \mid y_{n}) \log{a_{i, j}} \\
  \end{split}
\end{displaymath}

\begin{displaymath}
  \begin{split}
    \conde{ q_{n, t}^i q_{n, t+1}^j \log{a_{i,j}}}{y_{n}, l_{n}}
    & = l_{n, t} \cdot l_{n, t+1} \cdot q_{n, t}^i \cdot q_{n, t+1}^j \log{a_{i, j}}\\
    & + l_{n, t} \cdot \overline{l_{n, t+1}} \cdot q_{n, t}^i \cdot \gamma_{n, t+1}^j \log{a_{i, j}} \\
    & + \overline{l_{n, t}} \cdot l_{n, t+1} \cdot \gamma_{n, t}^i \cdot q_{n, t + 1}^j \log{a_{i, j}} \\
    & + \overline{l_{n, t}} \cdot \overline{l_{n, t+1}} \cdot \xi_{n, t}^{i,j} \log{a_{i, j}} \\
  \end{split}
\end{displaymath}

In the algorihtmic point of view, we have four cases, based on our
knowledge of the current and the next state.





% \begin{displaymath}
%   \begin{split}
%     \conde{q_{n, t}^i \log{ q_{n, t}^i q_{n, t+1}^j \log{a_{i,j}}}}{y_{n, t}, l_{n, t}}
%     & = l_{n, t} l_{n, t+1} q_{n, t}^i q_{n, t+1}^j \\
%     & + l_{n, t} (1 - l_{n, t+1}) q_{n, t}^i p(q_{n, t+1}^j = 1 \mid y) \\
%     & + (1 - l_{n, t}) l_{n, t+1} p(q_{n, t}^i = 1 \mid y) q_{n, t + 1}^j \\
%     & + (1 - l_{n, t}) (1 - l_{n, t+1}) p(q_{n, t}^i = 1 \mid y) p(q_{n, t + 1}^j = 1 \mid y) \\
%   \end{split}
% \end{displaymath}





\section{Semi-supervised HMM with mixed states}

We consider the semi-supervised framework. For each sequence we are
given observed values $y_0, \dots y_T$, and some of the hidden states
as well. Let us define $\mathcal{U}$ and $\mathcal{O}$ such that
$\mathcal{U} \cup \mathcal{O} = \{1, \dots T\}$.
\begin{displaymath}
  \mathcal{U} = \set{t \in \{1, \dots T\}}{\text{$q_t$ is not observed}}
\end{displaymath}
and
\begin{displaymath}
  \mathcal{O} \eqdef \{1, \dots T\} \setminus \mathcal{U} = \set{t \in \{1, \dots T\}}{\text{$q_t$ is observed}}
\end{displaymath}
Let us note $\{\tilde{q}_o \ssemi o \in \mathcal{O}\}$ the set of observed
values from the hidden states. Let us also define
\begin{displaymath}
  l_{t} \eqdef \mathbf{1}_\mathcal{O}(t) = \left\{
    \begin{array}{ll}
      1 &\text{ if $t \in \mathcal{O}$ }\\
      0 &\text{ otherwise }
    \end{array}
  \right.
\end{displaymath}
and
\begin{displaymath}
  \tilde{q}_{t} \eqdef \left\{
    \begin{array}{ll}
      \tilde{q}_t &\text{ if $t \in \mathcal{O}$ }\\
      0 &\text{ if $t \in \mathcal{U}$ }\\
    \end{array}
  \right.
\end{displaymath}
The HMM model is given below
\begin{displaymath}
  p(y, q) = p(y_0 \dots y_T, q_0 \dots q_T) = p(q_0) \prod_{t = 0}^{T-1} p(q_{t+1} \mid q_t) \prod_{t = 0}^{T} p(y_t \mid q_t)
\end{displaymath}

\subsection{Likelihood}

Writing the complete log likelihood gives
\begin{displaymath}
  \mathcal{L}(\theta \ssemi y, q) \eqdef \log{p(y, q \ssemi \theta)} = \log{p(q_0)} + \sum_{t = 0}^{T-1} \log{p(q_{t+1} \mid q_t)} + \sum_{t = 0}^{T} \log{ p(y_t \mid q_t \ssemi \theta) }
\end{displaymath}
With the one hot encoding notation, we have
\begin{displaymath}
  \mathcal{L}(\theta \ssemi y, q)
  = \sum_{i = 1}^{M} q_0^i \log{\pi_i}
  + \sum_{t = 0}^{T-1} \sum_{i = 1}^{M} \sum_{j = 1}^{M} q_t^i q_{t+1}^j \log{a_{i,j}}
  + \sum_{t = 0}^{T} \sum_{i = 1}^{M} q_t^i \log{ p(y_t \mid q_t^i \ssemi \theta_i) }
\end{displaymath}
We suppose that the emission probability of the HMM is an exponential
familly distribution. We write it as
\begin{displaymath}
  \label{exp-form}
  p(y_t \mid q_t^i \ssemi \theta_i) = h(y_t) \exp{\left(\theta_i^T T(y_t) - A(\theta_i)\right)}
\end{displaymath}
We then have
\begin{displaymath}
  \label{log-exp-form}
  \log{ p(y_t \mid q_t^i \ssemi \theta_i) } = \log{h(y_t)} + \theta_i^T T(y_t) - A(\theta_i)
\end{displaymath}
and we can rewrite the previous equation as
\begin{displaymath}
  \mathcal{L}(\theta \ssemi y, q)
  = \sum_{i = 1}^{M} q_0^i \log{\pi_i}
  + \sum_{t = 0}^{T-1} \sum_{i = 1}^{M} \sum_{j = 1}^{M} q_t^i q_{t+1}^j \log{a_{i,j}}
  + \sum_{t = 0}^{T} \sum_{i = 1}^{M} q_t^i \left( \log{h(y_t)} + \theta_i^T T(y_t) - A(\theta_i)\right)
\end{displaymath}
Because we want to maximize the likelihood with respect to the
parameters $(a_{i,j}, \pi_i, \theta_i)$, we want to maximize the
following expression
\begin{displaymath}
  \sum_{i = 1}^{M} q_0^i \log{\pi_i}
  + \sum_{t = 0}^{T-1} \sum_{i = 1}^{M} \sum_{j = 1}^{M} q_t^i q_{t+1}^j \log{a_{i,j}}
  + \sum_{t = 0}^{T} \sum_{i = 1}^{M} q_t^i \left(\theta_i^T T(y_t) - A(\theta_i)\right)
\end{displaymath}
The sufficient statistics for $a_{i,j}$ is given by
\begin{displaymath}
  m_{i, j} \eqdef \sum_{t = 0}^{T-1} q_t^i q_{t+1}^j
\end{displaymath}
The maximum likelihood estimate for $a_{i, j}$ is given by
\begin{displaymath}
  \hat{a}_{i,j} = \dfrac{
    m_{i,j}
  }{
    \sum_{k = 1}^M m_{i, k}
  }
\end{displaymath}
The sufficient statistics for $\theta_{i}$ is given by
\begin{displaymath}
  n_{i, j} \eqdef \sum_{t = 0}^{T-1} q_t^i T(y_t)
\end{displaymath}
Hence the maximum likelihood estimate for $\theta_{i, j}$ can be
recovered from
\begin{displaymath}
  \sum_{t = 0}^T q_t^i \grad_{\theta_i}A(\theta_i) = \sum_{t = 0}^T q_t^i T(y_t)
\end{displaymath}
i.e.
\begin{displaymath}
  \grad_{\theta_i}A(\theta_i) = \dfrac{
    \sum_{t = 0}^T q_t^i T(y_t)
  }{
    \sum_{t = 0}^T q_t^i
  }
\end{displaymath}


\textbf{Multinomial case}

For the multinomial case, we have
% \begin{displaymath}
%   \log{p(y ; \theta)}
%   = \sum_{j = 1}^{N-1} y_j\log(\theta_j)
%   + \left( 1 - \sum_{j = 1}^{N-1} y_{j} \right) \log \left( 1 - \sum_{j = 1}^{N-1} \theta_j \right)
% \end{displaymath}

\begin{displaymath}
  \left\{
  \begin{array}{rl}
    \log{h(y)} & = 0 \\[.5em]
    T(y) & = (y_1, y_{N}) \\[.5em]
    \theta & = (\eta_1, \eta_N) \\[.5em]
    A(\theta) &= \log{\left( \sum_{i = 1}^m \exp{\eta_i}\right)}
  \end{array}
  \right.
\end{displaymath}
with
\begin{displaymath}
  \eta_i = \dfrac{
    \pi_i
  }{
    \sum_{k = 1}^N \pi_k
  }
\end{displaymath}
We have
\begin{displaymath}
  \pd{A}{\eta_i}(\theta) = \dfrac{
    \exp{\eta_i}
  }{
    \sum_{i = 1}^m \exp{\eta_i}
  } = \pi_i
\end{displaymath}


\textbf{Multivariate gaussian case}

For the multivariate gaussian, we have
\begin{displaymath}
  \log{p(y ; \mu, \Sigma)} =
  - \dfrac{d}{2} \log(2 \pi)
  - \dfrac{1}{2} \log\det{\Sigma}
  - \dfrac{1}{2} (y - \mu)^T \Sigma^{-1} (y - \mu)
\end{displaymath}
Writing it in the exponential form above \eqref{log-exp-form}
\begin{displaymath}
  \log{ p(y \ssemi \theta) } = \log{h(y)} + \theta^T T(y) - A(\theta)
\end{displaymath}
We have
\begin{align*}
  \log{p(y ; \mu, \Sigma)}
  & =
    - \dfrac{d}{2} \log(2 \pi)
    - \dfrac{1}{2} \log\det{\Sigma}
    - \dfrac{1}{2} (y - \mu)^T \Sigma^{-1} (y - \mu)
  \\
  & =
    - \dfrac{d}{2} \log(2 \pi)
    - \dfrac{1}{2} \log\det{\Sigma}
    - \dfrac{1}{2} y^T \Sigma^{-1} y
    + \mu^T \Sigma^{-1} y
    - \dfrac{1}{2} \mu^T \Sigma^{-1} \mu
  \\
  & =
    - \dfrac{d}{2} \log(2 \pi)
    - \dfrac{1}{2} \log\det{\Sigma}
    - \dfrac{1}{2} \tr{\Sigma^{-1} y y^T}
    + \mu^T \Sigma^{-1} y
    - \dfrac{1}{2} \mu^T \Sigma^{-1} \mu
  \\
\end{align*}
We obtain
\begin{displaymath}
  \left\{
  \begin{array}{rl}
    \log{h(y)} & = - \dfrac{d}{2} \log(2 \pi) \\[.5em]
    T(y) & = (y, -\frac{1}{2} y y^T) \\[.5em]
    \theta & = (\mu^T \Sigma^{-1}, \Sigma^{-1}) \\[.5em]
    A(\theta) &= \dfrac{1}{2} \log\det{\Sigma} + \dfrac{1}{2} \mu^T \Sigma^{-1} \mu \\
  \end{array}
  \right.
\end{displaymath}
Let us reparametrize the multivariate gaussian using the canonical
parameters :
\begin{align*}
  \eta & =  \Sigma^{-1} \mu \\
  \Lambda & = \Sigma^{-1}
\end{align*}
\begin{displaymath}
  \left\{
  \begin{array}{rl}
    \log{h(y)} & = - \dfrac{d}{2} \log(2 \pi) \\[.5em]
    T(y) & = (y, -\frac{1}{2} y y^T) \\[.5em]
    \theta & = (\eta, \Lambda) \\[.5em]
    A(\theta) &= -\dfrac{1}{2} \log\det{\Lambda} + \dfrac{1}{2} \eta^T \Lambda^{-1} \eta \\
  \end{array}
  \right.
\end{displaymath}
We have
\begin{align*}
  \pd{A}{\eta}(\eta, \Lambda) & = \Lambda^{-1} \eta \\
  \pd{A}{\Lambda^{-1}}(\eta, \Lambda) & = \dfrac{1}{2} \Lambda + \dfrac{1}{2} \eta \eta^T
\end{align*}
From the previous equation, we have
\begin{displaymath}
  \Lambda^{-1} \eta = \dfrac{
    \sum_{t = 0}^T q_t^i y_t
  }{
    \sum_{t = 0}^T q_t^i
  }
\end{displaymath}
and
\begin{displaymath}
  \dfrac{1}{2} \Lambda + \dfrac{1}{2} \eta \eta^T  = -\dfrac{1}{2} \dfrac{
    \sum_{t = 0}^T q_t^i y_t y_t^T
  }{
    \sum_{t = 0}^T q_t^i
  }
\end{displaymath}
Hence
\begin{displaymath}
  \Lambda + \eta \eta^T  = - \dfrac{
    \sum_{t = 0}^T q_t^i y_t y_t^T
  }{
    \sum_{t = 0}^T q_t^i
  }
\end{displaymath}





% \begin{align*}
%   \log{h(y)} & = - \dfrac{d}{2} \log(2 \pi) \\
%   T(y) & = (y, y y^T) \\
%   \theta & = (- \mu^T \Sigma^{-1}, \Sigma^{-1}) \\
%   A(\theta) &= \dfrac{1}{2} \log\det{\Sigma^{-1}} - \dfrac{1}{2} \mu^T \Sigma^{-1} \mu
% \end{align*}







% \begin{displaymath}
%   \mathcal{L}(\theta \ssemi (y_1, q_1) \dots (y_N, q_N) ) = \sum_{n = 1}^N \sum_{i = 1}^{M} q_{n, 0}^i \log{\pi_i} + \sum_{n = 1}^N \sum_{t = 0}^{T-1} \sum_{i = 1}^{M} \sum_{j = 1}^{M} q_{n, t}^i q_{n, t+1}^j \log{a_{i,j}} + \sum_{n = 1}^N \sum_{t = 0}^{T} \sum_{i = 1}^{M} q_{n, t}^i \log{ p(y_{n, t} \mid q_{n, t}^i) }
% \end{displaymath}


% \begin{displaymath}
%   l_{n, t} = \left\{
%     \begin{array}{ll}
%       1 &\text{ if $q_{n, t}$ is observed } \\
%       0 &\text{ otherwise }
%     \end{array}
%     \right.
% \end{displaymath}


% \begin{displaymath}
%   \begin{split}
%     \mathcal{L}(\theta \ssemi (y_1, q_1) \dots (y_N, q_N) )
%     = &\sum_{n = 1}^N \sum_{i = 1}^{M} q_{n, 0}^i \log{\pi_i}
%     + \sum_{n = 1}^N \sum_{t = 0}^{T-1} \sum_{i = 1}^{M} \sum_{j = 1}^{M} q_{n, t}^i q_{n, t+1}^j \log{a_{i,j}} \\
%     & + \sum_{n = 1}^N \sum_{t = 0}^{T} \sum_{i = 1}^{M} q_{n, t}^i \log{ p(y_{n, t} \mid q_{n, t}^i \ssemi \theta_i) }
%   \end{split}
% \end{displaymath}

% \begin{displaymath}
%   \conde{q_{n, t}^i \log{ p(y_{n, t} \mid q_{n, t}^i)}}{y_{n, t}, l_{n, t}} =
%   \left(
%     l_{n, t} \cdot q_{n, t}^i + \overline{l_{n, t}} \cdot p(q_{n, t}^i = 1 \mid y)
%   \right)
%   \log{ p(y_{n, t} \mid q_{n, t}^i \ssemi \theta_i)}
% \end{displaymath}

% \begin{displaymath}
%   \begin{split}
%     \conde{ q_{n, t}^i q_{n, t+1}^j \log{a_{i,j}}}{y_{n}, l_{n}}
%     & = l_{n, t} \cdot l_{n, t+1} \cdot q_{n, t}^i \cdot q_{n, t+1}^j \log{a_{i, j}}\\
%     & + l_{n, t} \cdot \overline{l_{n, t+1}} \cdot q_{n, t}^i \cdot p(q_{n, t+1}^j = 1 \mid y_{n}) \log{a_{i, j}} \\
%     & + \overline{l_{n, t}} \cdot l_{n, t+1} \cdot p(q_{n, t}^i = 1 \mid y_{n}) \cdot q_{n, t + 1}^j \log{a_{i, j}} \\
%     & + \overline{l_{n, t}} \cdot \overline{l_{n, t+1}} \cdot p(q_{n, t}^i = 1 \wedge q_{n, t + 1}^j = 1  \mid y_{n}) \log{a_{i, j}} \\
%   \end{split}
% \end{displaymath}

% \begin{displaymath}
%   \begin{split}
%     \conde{ q_{n, t}^i q_{n, t+1}^j \log{a_{i,j}}}{y_{n}, l_{n}}
%     & = l_{n, t} \cdot l_{n, t+1} \cdot q_{n, t}^i \cdot q_{n, t+1}^j \log{a_{i, j}}\\
%     & + l_{n, t} \cdot \overline{l_{n, t+1}} \cdot q_{n, t}^i \cdot \gamma_{n, t+1}^j \log{a_{i, j}} \\
%     & + \overline{l_{n, t}} \cdot l_{n, t+1} \cdot \gamma_{n, t}^i \cdot q_{n, t + 1}^j \log{a_{i, j}} \\
%     & + \overline{l_{n, t}} \cdot \overline{l_{n, t+1}} \cdot \xi_{n, t}^{i,j} \log{a_{i, j}} \\
%   \end{split}
% \end{displaymath}

\subsection{Propagation belief}

We define
\begin{displaymath}
  \alpha(q_t) \eqdef p(q_{t}, y_0, \dots, y_{t+1})
\end{displaymath}
and
\begin{displaymath}
  \beta(q_{t}) \eqdef p(y_{t+1}, \dots y_T \mid q_t)
\end{displaymath}
By applying the sum product algorithm, we obtain We have the following
reccurence relations in the observed case
\begin{displaymath}
  \alpha(q_{t + 1}) = \sum_{q_t} \alpha(q_t) p(y_{t+1} \mid q_{t+1}) p(q_{t+1} \mid q_t) \delta(q_t, \overline{q_t})
\end{displaymath}
and
\begin{displaymath}
  \beta(q_{t}) = \sum_{q_{t+1}} \beta(q_{t+1}) p(y_{t+1} \mid q_{t+1}) p(q_{t+1} \mid q_t) \delta(q_{t+1}, \overline{q_{t+1}})
\end{displaymath}
Which gives the following update rules
\begin{displaymath}
  \alpha(q_{t + 1}) = \left\{
    \begin{array}{ll}
      \sum_{q_t} \alpha(q_t) p(y_{t+1} \mid q_{t+1}) p(q_{t+1} \mid q_t)
      & \text{ if $q_t$ is not observed} \\[0.5em]
      \alpha(\overline{q_t}) p(y_{t+1} \mid q_{t+1}) p(q_{t+1} \mid \overline{q_t})
      & \text{ if $q_t$ is observed} \\
    \end{array}
  \right.
\end{displaymath}
  and
\begin{displaymath}
  \beta(q_{t}) = \left\{
    \begin{array}{ll}
      \sum_{q_{t+1}} \beta(q_{t+1}) p(y_{t+1} \mid q_{t+1}) p(q_{t+1} \mid q_t)
      & \text{ if $q_{t+1}$ is not observed} \\[0.5em]
      \beta(\overline{q_{t+1}}) p(y_{t+1} \mid \overline{q_{t+1}}) p(\overline{q_{t+1}} \mid q_t)
      & \text{ if $q_{t+1}$ is observed} \\[0.5em]
    \end{array}
    \right.
\end{displaymath}



\begin{comment}
\begin{align*}
  \gamma(q_t)
  & \eqdef p(q_t \mid y_0, \dots, y_T, l_0, \dots, l_T) \\
  & = \sum_{q_{t+1}} p(q_t, q_{t+1} \mid y_0, \dots, y_T, l_0, \dots, l_T) \\
  & = \sum_{q_{t+1}} p(q_t \mid q_{t+1}, y_0, \dots, y_T, l_0, \dots, l_T) p(q_{t+1} \mid y_0, \dots, y_T, l_0, \dots, l_T) \\
  & = \sum_{q_{t+1}} p(q_t \mid q_{t+1}, y_0, \dots, y_T, l_0, \dots, l_T) \gamma_{t+1} \\
  & = \sum_{q_{t+1}} \dfrac{
    p(q_t, q_{t+1}, y_0, \dots, y_T, l_0, \dots, l_T)
    }{
    \sum_{q_t} p(q_t, q_{t+1}, y_0, \dots, y_T, l_0, \dots, l_T)
    } \gamma_{t+1} \\
  & = \sum_{q_{t+1}} \dfrac{
    p(q_{t+1} \mid q_t) p(q_t, y_0, \dots, y_T, l_0, \dots, l_T)
    }{
    \sum_{q_t} p(q_{t+1} \mid q_t) p(q_t, y_0, \dots, y_T, l_0, \dots, l_T)
    } \gamma_{t+1} \\
  & = \sum_{q_{t+1}} \dfrac{
    a_{q_t, {q_{t+1}}} \alpha(q_t)
    }{
    \sum_{q_t} a_{q_t, {q_{t+1}}} \alpha(q_t)
    } \gamma_{t+1}
\end{align*}


\begin{align*}
  \alpha(q_{t+1})
  & \eqdef p(q_{t+1},  y_0, \dots, y_T, l_0, \dots, l_T) \\
  & = \sum_{q_t} p(q_t, q_{t+1}, y_0, \dots, y_T, l_0, \dots, l_T) \\
  & = \sum_{q_t} p(q_{t+1} \mid q_t, y_0, \dots, y_T, l_0, \dots, l_T) p(q_t, y_0, \dots, y_T, l_0, \dots, l_T) \\
  & = \sum_{q_t} p(q_{t+1} \mid q_t) p(q_t, y_0, \dots, y_T, l_0, \dots, l_T) \\
\end{align*}
\end{comment}



% \begin{align*}
%   \alpha(q_{t + 1})
%   & \eqdef p(q_{t+1}, y_0, \dots, y_{t+1}) \\
%   & = \sum_{q_t} \alpha(q_t) p(y_{t+1} \mid q_{t+1}) p(q_{t+1} \mid q_t) \delta(q_t, \bar{q_t})
% \end{align*}

% \begin{align*}
%   \beta(q_{t})
%   & \eqdef p(y_{t+1}, \dots y_T \mid q_t) \\
%   & = \sum_{q_{t+1}} \beta(q_{t+1}) p(y_{t+1} \mid q_{t+1}) p(q_{t+1} \mid q_t) \delta(q_{t+1}, \bar{q_{t+1}})
      %   \end{align*}

\subsection{Emission distributions}

\begin{displaymath}
  p(y ; \mu, \Sigma) = \dfrac{1}{\sqrt{(2 \pi)^d \det{\Sigma}}}
  \exp{
    \left(
    -\dfrac{1}{2}
      (y - \mu)^T \Sigma^{-1} (y - \mu)
    \right)
  }
\end{displaymath}

\begin{displaymath}
  \log{p(y ; \mu, \Sigma)} =
  - \dfrac{d}{2} \log(2 \pi)
  - \dfrac{1}{2} \log\det{\Sigma}
  - \dfrac{1}{2} (y - \mu)^T \Sigma^{-1} (y - \mu)
\end{displaymath}


\begin{displaymath}
  \hat{\pi}_i^N = \dfrac{
    \sum_{n=1}^N \gamma_{0,n}^i
  }{
     \sum_{n=1}^N \sum_{i=1}^M \gamma_{0,n}^i
  } = \dfrac{
    \sum_{n=1}^N \gamma_{0,n}^i
  }{
     N
  }
\end{displaymath}


\begin{displaymath}
  \hat{a}_{i,j}^N = \dfrac{
    \sum_{n=1}^N \sum_{t=0}^{T_n-1} \xi(q_t^i, q_{t+1}^j)
  }{
    \sum_{n=1}^N \sum_{t=0}^{T_n-1} \gamma(q_t^i)
  }
\end{displaymath}


\begin{displaymath}
  \log{\hat{a}_{i,j}^N}
  = \log{\sum_{n=1}^N \sum_{t=0}^{T_n-1} \xi(q_t^i, q_{t+1}^j)}
  - \log{\sum_{n=1}^N \sum_{t=0}^{T_n-1} \gamma(q_t^i)}
\end{displaymath}


\begin{displaymath}
  \hat{\mu}_N = \dfrac{
    \sum_{n=1}^N \sum_{t=1}^{T_n} \gamma_{t,n} y_{t,n}
  }{
    \sum_{n=1}^N \sum_{t=1}^{T_n} \gamma_{t,n}
  }
\end{displaymath}


\begin{displaymath}
  \hat{\Sigma}_N = \dfrac{
    \sum_{n=1}^N \sum_{t=1}^{T_n}
    \gamma_{t,n} \left( y_{t,n} - \mu_N \right) \left( y_{t,n} - \mu_N \right)^T
  }{
    \sum_{n=1}^N \sum_{t=1}^{T_n} \gamma_{t,n}
  }
\end{displaymath}





\newpage
\bibliography{references}


\end{document}

%% \begin{figure}[p]
%%   \centering
%%   \begin{subfigure}[t]{0.40\textwidth}
%%     \centering
%%     \includegraphics[width=\textwidth]{LDA_classificationA_train.pdf}
%%     \caption{Training observations A ($150$ points)}\label{fig:LDA-A-train}
%%   \end{subfigure}
%%   \quad
%%   \begin{subfigure}[t]{0.40\textwidth}
%%     \centering
%%     \includegraphics[width=\textwidth]{LDA_classificationA_test.pdf}
%%     \caption{Test observations A ($1500$ points)}\label{fig:LDA-A-test}
%%   \end{subfigure}
%%   \vskip\baselineskip
%%   \begin{subfigure}[t]{0.40\textwidth}
%%     \centering
%%     \includegraphics[width=\textwidth]{LDA_classificationB_train.pdf}
%%     \caption{Training observations B ($150$ points)}\label{fig:LDA-B-train}
%%   \end{subfigure}
%%   \quad
%%   \begin{subfigure}[t]{0.40\textwidth}
%%     \centering
%%     \includegraphics[width=\textwidth]{LDA_classificationB_test.pdf}
%%     \caption{Test observations B ($1500$ points)}\label{fig:LDA-B-test}
%%   \end{subfigure}
%%   \vskip\baselineskip
%%   \begin{subfigure}[t]{0.40\textwidth}
%%     \centering
%%     \includegraphics[width=\textwidth]{LDA_classificationC_train.pdf}
%%     \caption{Training observations C ($150$ points)}\label{fig:LDA-C-train}
%%   \end{subfigure}
%%   \quad
%%   \begin{subfigure}[t]{0.40\textwidth}
%%     \centering
%%     \includegraphics[width=\textwidth]{LDA_classificationC_test.pdf}
%%     \caption{Test observations C ($1500$ points)}\label{fig:LDA-C-test}
%%   \end{subfigure}
%%   \caption{Sample data and decision boundary representation for the LDA classifier on the three files}\label{fig:LDA}
%% \end{figure}




  % \begin{table}[h!]
  %   \centering
  %   \begin{tabular}{|c|c|c|c||c|c|c|}
  %     \hline
  %     & \multicolumn{3}{c||}{\textbf{a0}} & \multicolumn{3}{c|}{\textbf{a1}}\\
  %     \hline
  %     & s0 & s1 & s2 & s0 & s1 & s2 \\
  %     \hline
  %     s0 & 0.45 & 0.00 & 0.55 & 0.00 & 0.00 & 1.00 \\
  %     s1 & 0.00 & 0.00 & 1.00 & 0.50 & 0.40 & 0.10 \\
  %     s2 & 0.60 & 0.00 & 0.40 & 0.00 & 0.90 & 0.10 \\
  %     \hline
  %   \end{tabular}
  %   \captionof{table}{Representation of the transition table
  %     corresponding to the graph} \label{tab:transition-table}
  % \end{table}

  % \begin{figure}[h]
  %   \centering
  %   \includegraphics[width=0.7\textwidth]{VI_convergence}
  %   \caption{Convergence of the value iteration algorithm}\label{fig:VI-convergence}
  % \end{figure}



    % \begin{tikzpicture}[baseline=(current bounding box.center),
    %   decoration=brace,
    %   large/.style={font=\large}]
    %   \matrix (M)[matrix of math nodes, nodes in empty cells,
    %   left delimiter={[}, right delimiter={]},
    %   column sep={0.6em,between origins},
    %   row sep={2.0em,between origins}
    %   ]{ &         & \\
    %     & y_t     & \\
    %     &         & \\
    %     & r_{t+1} & \\
    %   };
    %   \draw(M-3-1.west)--(M-3-3.east);
    %   % \draw(M-4-1.mid)--(M-4-9.mid);
    %   % \node[large] at (M-2-7){$0$};
    %   % \node[large] at (M-7-2){$0$};
    %   % \node[large] at (M-7-7){$0$};
    %   % \draw[decorate,transform canvas={xshift=0.3em, yshift=0.8em},thick] (M-1-1.mid west) -- node[above=2pt]{$n$}(M-1-3.mid east);
    %   % \draw[decorate,transform canvas={yshift=0.8em},thick] (M-1-1.west) -- node[above=1pt]{$d$}(M-1-3.east);
    %   % \draw[decorate,transform canvas={xshift=2.2em},thick] (M-7-5.north) -- node[right=1pt]{$n$}(M-11-5.south);
    %   % \draw[decorate,thick] (M-9-9.north) -- node[below=2pt]{$n$}(M-9-5.north);
    % \end{tikzpicture}
